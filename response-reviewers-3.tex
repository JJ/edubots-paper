% molde para cartas con membrete Dpto. y para sobres con ventana direcci¢n
\documentclass{letter}
\usepackage{graphicx}
\usepackage[spanish]{babel}      % castellano
\usepackage[latin1]{inputenc}    % tildes
\usepackage{url}

% para cartas en ingl‚s comentar la linea anterior e incluir la siguiente:
%\documentstyle[12pt]{a4letter}
% cabecera para hojas con membrete del Departamento y JJ
% Set left margin - The default is 1 inch, so the following 
% command sets a 0.5-inch left margin.
\setlength{\oddsidemargin}{0.0in}

% Set width of the text - What is left will be the right margin.
% In this case, right margin is 8.5in - 0.0in - 6.5in = 2.0in.
\setlength{\textwidth}{6.5in}

% Set top margin - The default is 1 inch, so the following 
% command sets a 0.5-inch top margin.
\setlength{\topmargin}{-1in}

% Set height of the text - What is left will be the bottom margin.
% In this case, bottom margin is 11in - 0.75in - 9.5in = 0.75in
\setlength{\textheight}{10in}

% cambiar estas definiciones para personalizar el documento
\def\yo{\sc Juan J. Merelo}
\def\email{\tt jmerelo@geneura.ugr.es}
\def\tel{34-958-243162}
\def\fax{34-958-248993}

% definiciones comunes
\def\granada{18071 Granada (Spain)}
\def\ucm{Universidad de Granada}
\def\fac{ETS Ingenierías Informática y Telecomucnicaciones}
\def\dpto{Departamento de Arquitectura y Tecnolog\'ia de Computadores}


%\input cabecera
\pagestyle{empty}

\begin{document}

\vspace{0.5cm}

\vspace{0.5cm}
\ \
% **** si no se quiere recuadro en la direcci¢n, comentar la linea siguiente
%\framebox[9cm][l]
{
\begin{minipage}[t]{9cm}
To whom it may concern
% ******* Poner aqu¡ nombre y direcci¢n del destinatario
\end{minipage}
}

\vspace{0.5cm}


% ********** comienzo del texto  *******************

{\bf Response to reviewers' comments}

We are very grateful to the editor and reviewers for their kind comments. We have rewritten
the paper quite extensively, improving grammar and, in some cases,
spelling. Additionally, we have revised references extensively, updating some
that were shown as ``under submission'' or ``accepted'', as well as fixed
capitalization in some others.

Next we will address the concerns of the single reviewer individually.

% ********** fin del texto  *******************

\hspace*{8cm} Yours,  \\


\hspace*{8cm} The authors


\newpage

%%%% Rev 3

{\it

  {\bf Reviewer \#1}:

The paper has been significantly revised to meet the comments submitted by various reviewers. As far as I am concerned the authors have satisfactorily addressed the comments of the earlier reviewers. As is often the case, the authors have made a case to answer queries, for example, about the research questions and the significance of the study.
While addressing the queries ought to lead to accepting the submission, the issue I have is that the added text - in blue font in my copy - has significant grammatical errors. For instance, just in the first additional paragraph there are a number of issues as shown below:

New technologies are being introduced and widely adopted into the classroom in the
last years. (in recent years).

However, in order to be successful, their application will require (many)
extra work and training for the teacher, as well as for the students. (delete many).

Thus, both of them will acquire new skills and, potentially, pupils will get a higher engagement to the subject, which is usually the main objective. (pupils will be more engaged).

Therefore the new text needs a careful proofread before the submission can be accepted.}

% -> answer 1

We have rewritten almost every paragraph of the paper, notably those that have
been pointed out in your review. We are again grateful for these suggestions.


\end{document}

